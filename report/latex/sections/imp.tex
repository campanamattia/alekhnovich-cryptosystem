\chapter{Alekhnovich's Cryptosystem}

\section{Overview}
This paper does not delve into the intricate details of the Alekhnovich algorithm or its potential advantages against quantum computers compared to existing algorithms. However, to provide context for further discussion, a brief overview of its operation is included.

\subsection{The Algorithm} The algorithm begins by generating two random matrices, $A\in \mathbb{Z}^{k\times n}_2$ and $S\in \mathbb{Z}^{l\times k}_2$. After, it generates a third random matrix, $E\in \mathbb{Z}^{l\times n}_2$. The rows of E are chosen from random and independent vectors of weight $t$ (the weight is computed as $\sqrt{n}$).The algorithm then defines the matrix $Y$ as $Y = SA + E$.


The matrices $Y$ and $A$ represent the public key of the algorithm. The public key is shared with anyone who wants to send encrypted messages to the sender.
The plain text to encrypt space is $m = \mathbf{C} \subset \{0,1\}^l $ where $\mathbf{C}$ is an error-correcting code. The error-correcting code implies an algorithm used to ensure that the message can be recovered even if it is corrupted by noise.
Now following steps are required for a secure exchange of information:
\begin{itemize}
 \item To encrypt the message, the sender performs the following steps:
    \begin{enumerate}
     \item The sender generates a random $t$-weight vector $e$ of $\mathbf{F}_2^n $.
     \item The sender multiplies $e^T$ by the public key $Y$ and $A$.
     \item The sender adds the message $m$ to the result of the multiplication $Ye^T$.
    \end{enumerate}
    The encrypted message is then given by the following vector: $$C(m) = (Ae^T, m+Ye^T)$$.
    \item To decrypt the message, the receiver performs the following steps:
    \begin{enumerate}
     \item The receiver multiplies the first part of the encrypted message, $Ae^T$, by the secret key $S$ getting $SAe^T = Ye^T-Ee^T $.
     \item The receiver subtracts from the second part of the encrypted message, $m+Ye^T$, the result of the previous multiplication.
    \end{enumerate}
    Now the receiver finds itself with a vector of the type: $$(m+Ye^T) - (Ye^T-Ee^T) = m + Ee^T$$
\end{itemize}
The algorithm works by exploiting the fact that the product of the matrix $E$ and the vector $e$ of weight $t$ is likely to be zero. This is due to the random positions of the few 1s in both the rows of $E$ and the vector $e$; the probability that even just one of these coincides is very low.
In the worst case, the code will distance itself from the original with a maximum Hamming distance of $t$, which is manageable with the error-correcting algorithm of the chosen code.

\subsection{Parameters}
The algorithm's implementation requires defining several parameters, including matrix dimensions and message length. To ensure the cryptosystem's reliability, we must impose a condition on the matrix dimensions: $$k + l < Rn$$ where $R$ is a constant less than 1. Once the message length ($l$ bits) is chosen, this rule guides the selection of suitable values for $k$ and $n$.

An important implementation factor is the practical limitation of allocating matrices with single-bit cells. A common solution is to select matrix dimensions that are divisible by the size of standard integers in the chosen programming language (e.g., \textit{uint64\_t} in C). 

Additionally, remember:
\begin{itemize}
    \item The weight of the vector $t$ is $$t = \sqrt{l}$$
    \item The transmission channel's error probability is $$p = \frac{t^2}{n}$$
\end{itemize}

\textit{For our implementation, we'll use the following parameters value: $l = 1300$ bits, $t = 144$ bits, $k = 2^{6}$ bits, $n = 2^{17}$ bits.}

%\section{Key Generation}

To facilitate secure message exchange, we must generate public and private keys. For efficient resource usage, we'll represent matrices with integers, treating individual bits as matrix cells. We'll employ the following libraries for robust randomness:
\begin{itemize}
    \item \textbf{librandombytes:} generates cryptography secure random seeds (\texttt{randombytes(x, X\_BYTES)} API, see \href{https://randombytes.cr.yp.to/index.html}{\underline{here}}).
    \item \textbf{xoshiro256 PRNG:} efficiently generates pseudo-random bits (look \href{https://prng.di.unimi.it}{\underline{here}} for the documentation).
\end{itemize}

\subsection{Matrix A}

Following the referenced work, matrix $A$ has dimensions $k \times n$ and contains random values. We'll delegate random value generation to trusted functions. To create $A$ a simple function will be used which, after initializing the seed, will generate as many random numbers as necessary:
$$
A \leftarrow \texttt{random\_matrix(a.rows, a.columns)}
$$ 
\textbf{Note:} If $n$ represents the total number of bits, we can reduce $A$'s size:
$a.columns \leftarrow \frac{n}{s}$, where $s$ is the \textit{size-of} the integer type chosen for our matrix.

\begin{algorithm}
\caption{\texttt{random\_matrix}}
\label{GRM}

\DontPrintSemicolon
\SetStartEndCondition{ }{}{}
\SetKwFunction{Range}{range}
\SetKw{KwTo}{in}
\SetKwFor{For}{for}{\string:}{}
\newcommand{\forrow}{$i$ \KwTo\Range{$rows$}}
\newcommand{\forcol}{$j$ \KwTo\Range{$columns$}}
\AlgoDontDisplayBlockMarkers\SetAlgoNoEnd\SetAlgoNoLine

\KwIn{rows, columns $\in \mathbb{N}$}
\KwOut{random matrix $\in \mathbb{N}^{r\times c}$}
\texttt{init\_seed()}\tcc*{Initializing the seed}
$M \leftarrow$ matrix\;
\For{\forrow}{
    \For{\forcol}{
        $m_{ij} \leftarrow$ \texttt{xoshiro256.next()}\tcc*{Filling the matrix}
    }
}
\Return{M}\;
\end{algorithm}

\subsection{Matrix S}
We follow the exact same procedure for matrix $S$, simply changing the input variables to our function:
$$
S\leftarrow\texttt{random\_matrix(s.rows, s.columns})
$$
As before, we reduce the column size to the ratio of the number of bits needed to the size of the instantiated matrix type.


\subsection{Matrix E}
This matrix has different requirements than the previous ones: a predetermined number of 1-bits (defined as weight $t$) must be randomly positioned within each row. To achieve this, we'll utilize a function to generate \texttt{e.rows} arrays, each containing a constant number of 1s, through a loop:
$$
E \leftarrow \texttt{weighted\_matrix(e.rows, e.columns, t)}
$$
Since we are addressing random positions within individual bits of an integer vector, it is pertinent to emphasize the processing of these integers that will form rows of the matrix:

\begin{algorithm}[H]
\caption{\texttt{weighted\_array}}
\label{WA}

\DontPrintSemicolon
\SetStartEndCondition{ }{}{}
\SetKwFunction{Range}{range}
\SetKw{KwTo}{in}
\SetKwFor{For}{for}{\string:}{}
\newcommand{\for}{$count$ \KwTo\Range{$t$}}
\AlgoDontDisplayBlockMarkers\SetAlgoNoEnd\SetAlgoNoLine

\KwIn{l, t $\in \mathbb{N}$}
\KwOut{random weighted vector $\in \mathbb{N}^{l}$}
$\mathbf{a} \leftarrow$ array\;
\For{\for}{
    \Repeat{no 1 has been inserted at that position yet}{
        $p \leftarrow$ \texttt{xoshiro256.next()}\;
        $i \leftarrow \text{p divided the size of the array's type}$\;
        $s \leftarrow \text{p module the size of the array's type}$\;
    }
    $a_i \leftarrow$ \textit{OR} with a 1 shifted by \(s\) bits\tcc*{Set the bit at calculated position}
}
\Return{$\mathbf{a}$}
\end{algorithm}


\subsection{Matrix Y}
Matrix \(Y\), in conjunction with the precomputed \(A\) matrix, constitutes the crucial public key for encrypting messages. The computation process involves:

\begin{enumerate}
    \item Multiplying the matrices \(S\) and \(A\).
    \item Adding the matrix \(E\) to the previously obtained result.
\end{enumerate}
$$
Y \leftarrow \texttt{compute\_pub(S, A, E)}
$$

Apart from the straightforward addition operation between matrices, it's essential to delve into the row-column product, especially when dealing with matrices where the fundamental elements are individual bits of integers rather than entire integer cells.

\paragraph{Transposition}
Initially, we must devise an efficient strategy for selecting the columns of bits in matrix A. One approach involves transposing the individual bits of the matrix, effectively converting its rows into columns for subsequent computations.

\begin{algorithm}[H]
\caption{\texttt{matrix\_transposed}}
\label{TM}

\DontPrintSemicolon
\SetStartEndCondition{ }{}{}
\SetKwFunction{Range}{range}
\SetKw{KwTo}{in}
\SetKwFor{For}{for}{\string:}{}
\newcommand{\forrow}{$i$ \KwTo\Range{$m.rows$}}
\newcommand{\forcol}{$j$ \KwTo\Range{$m.columns$}}
\newcommand{\forbit}{$k$ \KwTo\Range{$sizeof$}}
\AlgoDontDisplayBlockMarkers\SetAlgoNoEnd\SetAlgoNoLine

\KwIn{M $\in \mathbb{N}^{k \times n}$}
\KwOut{T $\in \mathbb{N}^{n \times h}$}
\KwData{by $size of$ we mean the type cell of the matrix}
$T \leftarrow$ matrix \tcc*{of n rows, and $\frac{\text{k}}{\text{size-of}}$ columns}
\For{\forrow}{
    \For{\forcol}{
        \For{\forbit}{
            $bit \leftarrow$ \texttt{fetch\_bit($m_{ij}$)} \tcc*{select the $k^{th}$ bit}
            $t_{yw} \leftarrow bit$ shifted by $i \pmod{sizeof}$ \tcc*{$y= (j * 64) + k,\quad w = \frac{i}{\text{size-of}}$}
        }
    }
}
\Return{T}\;
\end{algorithm}

\paragraph{Bit-wise AND $\&$ XOR}

Contextualizing the previous chapter, traditional arithmetic operations such as multiplication and addition are not applicable here for the row-by-column product. Instead, we perform bitwise operations on matrices. Initially, we conduct a bitwise AND operation between the cells of the two matrices, followed by an XOR operation on the results of these AND operations, as described below:

\begin{algorithm}[H]
\caption{\texttt{bax}}
\label{BAX}

\DontPrintSemicolon
\SetStartEndCondition{ }{}{}
\SetKwFunction{Range}{range}
\SetKw{KwTo}{in}
\SetKwFor{For}{for}{\string:}{}
\newcommand{\forlength}{$i$ \KwTo\Range{$l$}}
\AlgoDontDisplayBlockMarkers\SetAlgoNoEnd\SetAlgoNoLine

\KwIn{$\mathbf{a,v} \in \mathbb{N}^l$}
\KwOut{result $\in \mathbb{N}_2$}

$result \leftarrow 0$\;
\For{\forlength}{
$and \leftarrow a_i\& v_i$\tcc*{bit-wise AND operation}
$result \leftarrow result$ + \texttt{count\_ones($and$)}
}
$result \leftarrow result\pmod{2}$ \tcc*{simplified XOR}
\Return{result};
\end{algorithm}
where the $\mathbf{a}$ and $\mathbf{v}$ arrays represent the $S$ and $A^T$ single row.

\section{Key Generation}

To facilitate secure message exchange, we must generate public and private keys. For efficient resource usage, we'll represent matrices with integers, treating individual bits as matrix cells. We'll employ the following libraries for robust randomness:
\begin{itemize}
    \item \textbf{librandombytes:} generates cryptography secure random seeds (\texttt{randombytes(x, X\_BYTES)} API, see \href{https://randombytes.cr.yp.to/index.html}{\underline{here}}).
    \item \textbf{xoshiro256 PRNG:} efficiently generates pseudo-random bits (look \href{https://prng.di.unimi.it}{\underline{here}} for the documentation).
\end{itemize}

\begin{algorithm}[H]
\caption{\texttt{init\_seed}}
\label{IS}

\DontPrintSemicolon
\SetStartEndCondition{ }{}{}
\SetKwFunction{Range}{range}
\SetKw{KwTo}{in}
\SetKwFor{For}{for}{\string:}{}
\newcommand{\fori}{$i$ \KwTo\Range{$4$}}
\newcommand{\forj}{$j$ \KwTo\Range{$8$}}
\AlgoDontDisplayBlockMarkers\SetAlgoNoEnd\SetAlgoNoLine

\KwData{vector $\mathbf{s}$ is the seed for \texttt{xosihiro.next()}}
\For{\fori}{
    $\mathbf{seed} \leftarrow$ array\;
    \texttt{randombytes(seed, 8)}\;
    \For{\forj}{
        $s_i \leftarrow s_i \ll 8$ \;
        $s_i |= seed_j$\;
    }
}
\end{algorithm}

\subsection{Matrix A}

Following the referenced work, matrix $A$ has dimensions $k \times n$ and contains random values. We'll delegate random value generation to trusted functions. To create $A$ a simple function will be used which, after initializing the seed, will generate as many random numbers as necessary:
$$
A \leftarrow \texttt{random\_matrix(a.rows, a.columns)}
$$ 
\textbf{Note:} If $n$ represents the total number of bits, we can reduce $A$'s size:
$a.columns \leftarrow \frac{n}{s}$, where $s$ is the \textit{size-of} the integer type chosen for our matrix.

\begin{algorithm}[H]
\caption{\texttt{random\_matrix}}
\label{GRM}

\DontPrintSemicolon
\SetStartEndCondition{ }{}{}
\SetKwFunction{Range}{range}
\SetKw{KwTo}{in}
\SetKwFor{For}{for}{\string:}{}
\newcommand{\forrow}{$i$ \KwTo\Range{$rows$}}
\newcommand{\forcol}{$j$ \KwTo\Range{$columns$}}
\AlgoDontDisplayBlockMarkers\SetAlgoNoEnd\SetAlgoNoLine

\KwIn{rows, columns $\in \mathbb{N}$}
\KwOut{random matrix $\in \mathbb{Z}_2^{r\times c}$}
\texttt{init\_seed()}\tcc*{Initializing the seed}
$M \leftarrow$ matrix\;
\For{\forrow}{
    \For{\forcol}{
        $m_{ij} \leftarrow$ \texttt{xoshiro256.next()}\tcc*{Filling the matrix}
    }
}
\Return{M}\;
\end{algorithm}

\subsection{Matrix S}
We follow the exact same procedure for matrix $S$, simply changing the input variables to our function:
$$
S\leftarrow\texttt{random\_matrix(s.rows, s.columns})
$$
As before, we reduce the column size to the ratio of the number of bits needed to the size of the instantiated matrix type.


\subsection{Matrix E}
This matrix has different requirements than the previous ones: a predetermined number of 1-bits (defined as weight $t$) must be randomly positioned within each row. To achieve this, we'll utilize a function to generate \texttt{e.rows} arrays, each containing a constant number of 1s, through a loop:
$$
E \leftarrow \texttt{weighted\_matrix(e.rows, e.columns, t)}
$$
Since we are addressing random positions within individual bits of an integer vector, it is pertinent to emphasize the processing of these integers that will form rows of the matrix:

\begin{algorithm}[H]
\caption{\texttt{weighted\_array}}
\label{WA}

\DontPrintSemicolon
\SetStartEndCondition{ }{}{}
\SetKwFunction{Range}{range}
\SetKw{KwTo}{in}
\SetKwFor{For}{for}{\string:}{}
\newcommand{\for}{$count$ \KwTo\Range{$t$}}
\AlgoDontDisplayBlockMarkers\SetAlgoNoEnd\SetAlgoNoLine

\KwIn{l, t $\in \mathbb{N}$}
\KwOut{random weighted vector $\in \mathbb{F}_2^{l}$}
\KwData{by $size of$ we mean the type cell of the matrix}
\texttt{init\_seed()}\;
$\mathbf{a} \leftarrow$ array\;
\For{\for}{
    \Repeat{$\text{fecth\_bit}(a_i, s) \not= 1$}{
        $p \leftarrow$ \texttt{xoshiro256.next()}\;
        $i \leftarrow \frac{p}{sizeof}$ \tcc*{Casted to integer}
        $s \leftarrow p \pmod{sizeof}$\;
    }
    $a_i |= \texttt{shift\_bit(1, s)}$ \tcc*{Set the bit at calculated position}
}
\Return{a}
\end{algorithm}


\subsection{Matrix Y}
Matrix \(Y\), in conjunction with the precomputed \(A\) matrix, constitutes the crucial public key for encrypting messages. The computation process involves:

\begin{enumerate}
    \item Multiplying the matrices \(S\) and \(A\).
    \item Adding the matrix \(E\) to the previously obtained result.
\end{enumerate}
$$
Y \leftarrow \texttt{compute\_pub(S, A, E)}
$$

Apart from the straightforward addition operation between matrices, it's essential to delve into the row-column product, especially when dealing with matrices where the fundamental elements are individual bits of integers rather than entire integer cells.

\paragraph{Transposition}
Initially, we must devise an efficient strategy for selecting the columns of bits in matrix A. One approach involves transposing the individual bits of the matrix, effectively converting its rows into columns for subsequent computations.

\begin{algorithm}[H]
\caption{\texttt{matrix\_transposed}}
\label{TM}

\DontPrintSemicolon
\SetStartEndCondition{ }{}{}
\SetKwFunction{Range}{range}
\SetKw{KwTo}{in}
\SetKwFor{For}{for}{\string:}{}
\newcommand{\forrow}{$i$ \KwTo\Range{$m.rows$}}
\newcommand{\forcol}{$j$ \KwTo\Range{$m.columns$}}
\newcommand{\forbit}{$k$ \KwTo\Range{$sizeof$}}
\AlgoDontDisplayBlockMarkers\SetAlgoNoEnd\SetAlgoNoLine

\KwIn{M $\in \mathbb{Z}_2^{k \times n}$}
\KwOut{T $\in \mathbb{Z}_2^{n \times h}$}
\KwData{by $size of$ we mean the type cell of the matrix}
$T \leftarrow$ matrix\;
\For{\forrow}{
    \For{\forcol}{
        \For{\forbit}{
            $bit \leftarrow$ \texttt{fetch\_bit($m_{ij}, k$)}\;
            $t_{yw} |=$ \texttt{shift\_bit($bit, sizeof - i\pmod{sizeof}$)}\;
        }
    }
}
\Return{T}\;
\end{algorithm}

where \texttt{fetch\_bit($e, k$)} returns the bit in position $k^{th}$ of $e$ and \texttt{shift\_bit(b, s)} the bit $b$ shifted left by $s$ positions.

\paragraph{Bit-wise AND $\&$ XOR}

Contextualizing the previous chapter, traditional arithmetic operations such as multiplication and addition are not applicable here for the row-by-column product. Instead, we perform bitwise operations on matrices. Initially, we conduct a bitwise AND operation between the cells of the two matrices, followed by an XOR operation on the results of these AND operations, as described below:

\begin{algorithm}[H]
\caption{\texttt{bax}}
\label{BAX}

\DontPrintSemicolon
\SetStartEndCondition{ }{}{}
\SetKwFunction{Range}{range}
\SetKw{KwTo}{in}
\SetKwFor{For}{for}{\string:}{}
\newcommand{\forlength}{$i$ \KwTo\Range{$l$}}
\AlgoDontDisplayBlockMarkers\SetAlgoNoEnd\SetAlgoNoLine

\KwIn{$\mathbf{a,v} \in \mathbb{Z}_2^l$}
\KwOut{result $\in \mathbb{Z}_2$}
\KwData{$\mathbf{a}$ and $\mathbf{v}$ arrays represent the $S$ and $A^T$ single row}

$result \leftarrow 0$\;
\For{\forlength}{
$and \leftarrow a_i\& v_i$\tcc*{Bit-wise AND operation}
$result \leftarrow result^\land and$\;
}
$result \leftarrow \texttt{count\_ones(result)}\pmod{2}$ \tcc*{Simplified XOR}
\Return{result};
\end{algorithm}

%\section{Encryption}

Following the generation of cryptographic keys, they are utilized for encryption and decryption purposes. In the transmission of encrypted messages, apart from selecting an appropriate element from our cryptographic system, denoted as $m = \mathbf{C} \subset \{0,1\}^l $, it is imperative to generate a random weighted vector, maintaining the previous weight $t$. This can be achieved through the function introduced earlier:
$$
\mathbf{e}\leftarrow\texttt{weighted\_array(l, t)}
$$

For the encryption procedure itself, we can rely on the previously introduced functions. A call to two functions is sufficient:
\begin{enumerate}
    \item $\mathbf{nnc} \leftarrow\texttt{compute\_nonce(A, e)}$, computing the row-by-column product (with a "matrix" consisting of a single column)
    \item $\mathbf{enc} \leftarrow\texttt{cipher(Y, e, m)}$, which, in addition to performing the row-by-column product like the one above, also adds our message $m$ to the result
\end{enumerate}
$$
\mathbf{message} \leftarrow(\texttt{nnc, enc})
$$
\section{Encryption}

Following the generation of cryptographic keys, they are utilized for encryption and decryption purposes. In the transmission of encrypted messages, apart from selecting an appropriate element from our cryptographic system, denoted as $m = \mathbf{C} \subset \{0,1\}^l $, it is imperative to generate a random weighted vector, maintaining the previous weight $t$. This can be achieved through the function introduced earlier:
$$
\mathbf{e}\leftarrow\texttt{weighted\_array(n, t)}
$$

For the encryption procedure itself, we can rely on the previously introduced functions. A call to two functions is sufficient:
\begin{enumerate}
    \item $\mathbf{nnc} \leftarrow\texttt{compute\_nonce(A, e)}$\footnote{A nonce is a unique, one-time-use value used to ensure secure communication.}, computing the row-by-column product (with a "matrix" consisting of a single column)
    \item $\mathbf{cmp} \leftarrow\texttt{cipher(Y, e, m)}$, which, in addition to performing the row-by-column product like the one above, also adds our message $m$ to the result
\end{enumerate}
$$
\mathbf{packet} \leftarrow\texttt{encryption(m, A, Y)}
$$

\begin{algorithm}[H]
\caption{\texttt{encryption}}
\label{enc}

\DontPrintSemicolon
\SetStartEndCondition{ }{}{}
\AlgoDontDisplayBlockMarkers\SetAlgoNoEnd\SetAlgoNoLine

\KwIn{$\mathbf{m} \in \mathbf{C}$, $A \in \mathbf{Z}_2^{k\times n}$, $Y \in \mathbf{Z}_2^{l \times n}$}
\KwOut{$\mathbf{encrypted} \in \mathbb{Z}_2^{k+l}$}
$\mathbf{e} \leftarrow \texttt{weighted\_array(n, t)}$\;
$\mathbf{nnc} \leftarrow \texttt{row\_column(A, e)}$\;
$\mathbf{cmp} \leftarrow \texttt{row\_column(Y, e)}$\;
$\mathbf{cmp} \leftarrow \texttt{sum\_array(cmp, m)}$\;
$\mathbf{encrypted} \leftarrow \texttt{concat(nnc, cmp)}$ \tcc*{concatenate the two arrays}
\Return{encrypted};
\end{algorithm}

%\section{Decryption}
Upon receiving an encrypted message, it is necessary to separate the nonce from the message and utilize the first part of the message to compute the key required for decrypting this individual message. Once we have obtained the first part of the message, we multiply it by our private key $S$ to obtain a result as follows:
$$
\mathbf{key}=SAe^T\leftarrow \texttt{row\_column(S, nnc)}
$$
And from that we can use the \(\mathbf{k}\) vector as: $Y\mathbf{e}^T=\mathbf{k} + E\mathbf{e}^T\rightarrow \mathbf{k} = Y\mathbf{e}^T - E\mathbf{e}^T$

At this point, all that remains is to subtract the vector $\mathbf{k}$ obtained above from our $\mathbf{code}$ to achieve a result of the form:
$$
\mathbf{m}\sim \mathbf{dec}\leftarrow \mathbf{enc} - \mathbf{k}
$$

Although it would be desirable to claim that by doing so we obtain our decrypted message, we still need to address the issue by $E\mathbf{e}^T$, which we tackle in the subsequent section dedicated to error correction in codes.
\section{Decryption}
Upon receiving an encrypted message, it is necessary to separate the nonce from the message and utilize the first part of the message to compute the key required for decrypting this individual message. Once we have obtained the first part of the message, we multiply it by our private key $S$ to obtain a result as follows:
$$
SA\mathbf{e}^T\leftarrow \texttt{row\_column(S, nnc)}
$$
And from that we can use the \(\mathbf{key} = SAe^T\) vector as: $\mathbf{key} = Y\mathbf{e}^T - E\mathbf{e}^T$

At this point, all that remains is to subtract the vector $\mathbf{key}$ obtained above from our code to achieve a result of the form:
$
\mathbf{cmp} - \mathbf{key} = \mathbf{m} + Y\mathbf{e}^T - Y\mathbf{e}^T +  E\mathbf{e}^T
$

$$
\mathbf{message}\leftarrow\texttt{decryption(paket, S)}
$$

\begin{algorithm}[H]
\caption{\texttt{decryption}}
\label{dec}

\DontPrintSemicolon
\SetStartEndCondition{ }{}{}
\AlgoDontDisplayBlockMarkers\SetAlgoNoEnd\SetAlgoNoLine

\KwIn{$\mathbf{packet} \in \mathbf{Z}_2^{k+l}$, $S \in \mathbf{Z}_2^{l\times n}$}
\KwOut{$\mathbf{dec} \in \mathbb{Z}_2^l$}
$\mathbf{key} \leftarrow \texttt{row\_column(S, packet[..k])}$ \tcc*{packet[..k] position from 0 to k-1}
$\mathbf{dec} \leftarrow \texttt{sub\_array(packet[k..], key)}$ \tcc*{packet[k..] position from k to end}
\Return{dec};
\end{algorithm}

Although it would be desirable to claim that by doing so we obtain our decrypted message, we still need to address the issue by $E\mathbf{e}^T$, which we tackle in the subsequent section dedicated to error correction in codes.



%input{sec_imp/error_code}
\section{Error Code}
At this juncture, as previously mentioned, we encounter a code of the form $\mathbf{m}+E\mathbf{e}^T$. While seemingly distant from our original message, given the construction of $\mathbf{m}$ as a code equipped with an error correction algorithm, we can identify the few variations induced by $E\mathbf{e}^T$.

One of the most practical solutions is to transmit the same message multiple times, each time with a different nonce, and then decrypt it to identify which bit appears most frequently.
