\chapter{Future Developments}

Public-key cryptography employs a pair of keys: a public key, which is accessible to everyone, and a private key, kept confidential. The public key allows anyone to encrypt messages, but only the holder of the private key can decrypt them. This approach simplifies secure communication by eliminating the need for prior secure key exchange, allowing secure interactions through the exchange of public keys.

Quantum computing introduces qubits, which can exist in multiple states simultaneously due to the phenomenon of superposition. This property, combined with entanglement, significantly enhances the computational power of quantum systems. The Quantum Fourier Transform (QFT), a quantum analog of the classical Fourier transform, is particularly important in quantum algorithms for identifying periodicity in superpositions.

RSA cryptography relies on the difficulty of factoring large composite numbers. Specifically, RSA security is based on the challenge of determining the prime factors of a large number \( N \), which is the product of two prime numbers. The RSA algorithm uses these prime factors to generate public and private keys, with the system's security depending on the infeasibility of factorizing \( N \) within a reasonable time frame using classical computers.

While classical factorization methods, such as the General Number Field Sieve algorithm, exist, their practical application is limited by significant computational requirements. Quantum computers, however, can execute Shor's algorithm, which efficiently solves the integer factorization problem. The Quantum Fourier Transform is integral to this process, as it identifies the periodicity in functions corresponding to the factors of \( N \), thus enabling the determination of the private key.

Given the rapid advancements in quantum computing and the decreasing qubit requirements for practical quantum algorithms, there is an urgent need to develop post-quantum cryptographic solutions. These solutions aim to secure digital communications against the capabilities of quantum adversaries, ensuring data integrity and confidentiality in a post-quantum world. The development and standardization of such cryptographic methods are critical to maintaining robust security frameworks in the face of emerging quantum threats.
