\section{Key Generation}

To facilitate secure message exchange, we must generate public and private keys. For efficient resource usage, we'll represent matrices with integers, treating individual bits as matrix cells. We'll employ the following libraries for robust randomness:
\begin{itemize}
    \item \textbf{librandombytes:} generates cryptography secure random seeds (\texttt{randombytes(x, X\_BYTES)} API, see \href{https://randombytes.cr.yp.to/index.html}{\underline{here}}).
    \item \textbf{xoshiro256 PRNG:} efficiently generates pseudo-random bits (look \href{https://prng.di.unimi.it}{\underline{here}} for the documentation).
\end{itemize}

\subsection{Matrix A}

Following the referenced work, matrix $A$ has dimensions $k \times n$ and contains random values. We'll delegate random value generation to trusted functions. To create $A$ a simple function will be used which, after initializing the seed, will generate as many random numbers as necessary:
$$
A \leftarrow \texttt{random\_matrix(a.rows, a.columns)}
$$ 
\textbf{Note:} If $n$ represents the total number of bits, we can reduce $A$'s size:
$a.columns \leftarrow \frac{n}{s}$, where $s$ is the \textit{size-of} the integer type chosen for our matrix.

\begin{algorithm}
\caption{\texttt{random\_matrix}}
\label{GRM}

\DontPrintSemicolon
\SetStartEndCondition{ }{}{}
\SetKwFunction{Range}{range}
\SetKw{KwTo}{in}
\SetKwFor{For}{for}{\string:}{}
\newcommand{\forrow}{$i$ \KwTo\Range{$rows$}}
\newcommand{\forcol}{$j$ \KwTo\Range{$columns$}}
\AlgoDontDisplayBlockMarkers\SetAlgoNoEnd\SetAlgoNoLine

\KwIn{rows, columns $\in \mathbb{N}$}
\KwOut{random matrix $\in \mathbb{N}^{r\times c}$}
\texttt{init\_seed()}\tcc*{Initializing the seed}
$M \leftarrow$ matrix\;
\For{\forrow}{
    \For{\forcol}{
        $m_{ij} \leftarrow$ \texttt{xoshiro256.next()}\tcc*{Filling the matrix}
    }
}
\Return{M}\;
\end{algorithm}

\subsection{Matrix S}
We follow the exact same procedure for matrix $S$, simply changing the input variables to our function:
$$
S\leftarrow\texttt{random\_matrix(s.rows, s.columns})
$$
As before, we reduce the column size to the ratio of the number of bits needed to the size of the instantiated matrix type.


\subsection{Matrix E}
This matrix has different requirements than the previous ones: a predetermined number of 1-bits (defined as weight $t$) must be randomly positioned within each row. To achieve this, we'll utilize a function to generate \texttt{e.rows} arrays, each containing a constant number of 1s, through a loop:
$$
E \leftarrow \texttt{weighted\_matrix(e.rows, e.columns, t)}
$$
Since we are addressing random positions within individual bits of an integer vector, it is pertinent to emphasize the processing of these integers that will form rows of the matrix:

\begin{algorithm}[H]
\caption{\texttt{weighted\_array}}
\label{WA}

\DontPrintSemicolon
\SetStartEndCondition{ }{}{}
\SetKwFunction{Range}{range}
\SetKw{KwTo}{in}
\SetKwFor{For}{for}{\string:}{}
\newcommand{\for}{$count$ \KwTo\Range{$t$}}
\AlgoDontDisplayBlockMarkers\SetAlgoNoEnd\SetAlgoNoLine

\KwIn{l, t $\in \mathbb{N}$}
\KwOut{random weighted vector $\in \mathbb{N}^{l}$}
$\mathbf{a} \leftarrow$ array\;
\For{\for}{
    \Repeat{no 1 has been inserted at that position yet}{
        $p \leftarrow$ \texttt{xoshiro256.next()}\;
        $i \leftarrow \text{p divided the size of the array's type}$\;
        $s \leftarrow \text{p module the size of the array's type}$\;
    }
    $a_i \leftarrow$ \textit{OR} with a 1 shifted by \(s\) bits\tcc*{Set the bit at calculated position}
}
\Return{$\mathbf{a}$}
\end{algorithm}


\subsection{Matrix Y}
Matrix \(Y\), in conjunction with the precomputed \(A\) matrix, constitutes the crucial public key for encrypting messages. The computation process involves:

\begin{enumerate}
    \item Multiplying the matrices \(S\) and \(A\).
    \item Adding the matrix \(E\) to the previously obtained result.
\end{enumerate}
$$
Y \leftarrow \texttt{compute\_pub(S, A, E)}
$$

Apart from the straightforward addition operation between matrices, it's essential to delve into the row-column product, especially when dealing with matrices where the fundamental elements are individual bits of integers rather than entire integer cells.

\paragraph{Transposition}
Initially, we must devise an efficient strategy for selecting the columns of bits in matrix A. One approach involves transposing the individual bits of the matrix, effectively converting its rows into columns for subsequent computations.

\begin{algorithm}[H]
\caption{\texttt{matrix\_transposed}}
\label{TM}

\DontPrintSemicolon
\SetStartEndCondition{ }{}{}
\SetKwFunction{Range}{range}
\SetKw{KwTo}{in}
\SetKwFor{For}{for}{\string:}{}
\newcommand{\forrow}{$i$ \KwTo\Range{$m.rows$}}
\newcommand{\forcol}{$j$ \KwTo\Range{$m.columns$}}
\newcommand{\forbit}{$k$ \KwTo\Range{$sizeof$}}
\AlgoDontDisplayBlockMarkers\SetAlgoNoEnd\SetAlgoNoLine

\KwIn{M $\in \mathbb{N}^{k \times n}$}
\KwOut{T $\in \mathbb{N}^{n \times h}$}
\KwData{by $size of$ we mean the type cell of the matrix}
$T \leftarrow$ matrix \tcc*{of n rows, and $\frac{\text{k}}{\text{size-of}}$ columns}
\For{\forrow}{
    \For{\forcol}{
        \For{\forbit}{
            $bit \leftarrow$ \texttt{fetch\_bit($m_{ij}$)} \tcc*{select the $k^{th}$ bit}
            $t_{yw} \leftarrow bit$ shifted by $i \pmod{sizeof}$ \tcc*{$y= (j * 64) + k,\quad w = \frac{i}{\text{size-of}}$}
        }
    }
}
\Return{T}\;
\end{algorithm}

\paragraph{Bit-wise AND $\&$ XOR}

Contextualizing the previous chapter, traditional arithmetic operations such as multiplication and addition are not applicable here for the row-by-column product. Instead, we perform bitwise operations on matrices. Initially, we conduct a bitwise AND operation between the cells of the two matrices, followed by an XOR operation on the results of these AND operations, as described below:

\begin{algorithm}[H]
\caption{\texttt{bax}}
\label{BAX}

\DontPrintSemicolon
\SetStartEndCondition{ }{}{}
\SetKwFunction{Range}{range}
\SetKw{KwTo}{in}
\SetKwFor{For}{for}{\string:}{}
\newcommand{\forlength}{$i$ \KwTo\Range{$l$}}
\AlgoDontDisplayBlockMarkers\SetAlgoNoEnd\SetAlgoNoLine

\KwIn{$\mathbf{a,v} \in \mathbb{N}^l$}
\KwOut{result $\in \mathbb{N}_2$}

$result \leftarrow 0$\;
\For{\forlength}{
$and \leftarrow a_i\& v_i$\tcc*{bit-wise AND operation}
$result \leftarrow result$ + \texttt{count\_ones($and$)}
}
$result \leftarrow result\pmod{2}$ \tcc*{simplified XOR}
\Return{result};
\end{algorithm}
where the $\mathbf{a}$ and $\mathbf{v}$ arrays represent the $S$ and $A^T$ single row.
