\chapter{Project Documentation}

This chapter provides an overview of the C files included in the project. The files documented here are:
\begin{itemize}
    \item \texttt{test.c}
    \item \texttt{backend.c}
    \item \texttt{api.c}
    \item \texttt{main.c}
\end{itemize}
Each file plays a specific role in the overall project structure, and this chapter will detail their functionality.

\section{\texttt{main.c}}
The \texttt{main.c} file serves as the entry point for the program. It processes command-line arguments to determine which operation the program should perform, such as testing, key generation, encryption, decryption, or error correction.

\subsection{Key Functions}
\begin{itemize}
    \item \texttt{main(int argc, char *argv[])}: It parses the command-line arguments, determines the appropriate command to execute, and invokes the corresponding function.
    \item \texttt{Command get\_command(const char *command)}: It maps a command string to its corresponding enumeration value.
    \item \texttt{void print\_err(const char *name, const char *msg)}
\end{itemize}

\subsection{Details}
The \texttt{main.c} file handles the main logic of the application by evaluating the command-line input and executing the appropriate operation based on the command provided. The supported commands include:

\begin{itemize}
    \item \textbf{TEST}: Runs the test suite by invoking the \texttt{test()} function from the \texttt{test.c} file.
    \item \textbf{GENERATE}: Calls the \texttt{generate\_key()} function to generate encryption keys.
    \item \textbf{ENCRYPT}: Encrypts a message using the provided key paths, handled by the \texttt{encrypt()} function.
    \item \textbf{DECRYPT}: Decrypts a message using the provided file paths, handled by the \texttt{decrypt()} function.
    \item \textbf{CORRECT}: Corrects data using input files, handled by the \texttt{correct()} function.
\end{itemize}

The control flow in the \texttt{main()} function is structured as follows:

\begin{enumerate}
    \item Argument Validation: checks if at least one command-line argument is provided.
    \item Command Identification: is used to identify the command type based on the first argument.
    \item Command Execution: the program invokes the appropriate function. If insufficient arguments are provided for a specific command, an error message is printed using \texttt{print\_err()}.
    \item Error Handling: if the command is invalid, an error message is displayed, and the program exits with a status code.
\end{enumerate}

The \texttt{main.c} file integrates closely with other modules like \texttt{api.c} and \texttt{test.c}, enabling it to act as a central coordinator for different operations in the application. The use of an enumeration for commands makes it easy to extend the program with additional functionalities in the future.

\section{\texttt{test.c}}

The \texttt{test.c} file contains two key functions for testing the cryptographic processes: \texttt{test()} and \texttt{shortcut()}. These functions serve different purposes and offer distinct levels of testing and validation.

\subsection{\texttt{test() Function}}
The \texttt{test()} function is the primary testing function that performs a comprehensive validation of the cryptographic operations. It is invoked with an integer parameter \texttt{mod} to control its behavior.

\begin{itemize}
    \item Comprehensive Testing: When \texttt{mod} is set to `0`, the \texttt{test()} function conducts a full testing process, which includes generating random matrices, writing and reading keys, encrypting and decrypting a random message, and verifying the integrity of the entire process.
    \item Matrix Operations: The function generates matrices required for encryption and decryption, ensuring that the cryptographic algorithms work correctly with dynamically generated data.
    \item Resource Management: It manages resources meticulously, allocating and freeing memory for matrices and arrays to prevent memory leaks.
    \item Validation: The function checks that the generated keys can be correctly read back from the files and that the encryption and decryption processes produce the expected results.
\end{itemize}

The \texttt{test()} function is designed for a deep and thorough validation of the entire cryptographic process. It is used to ensure that all components work together as expected under typical conditions.

\subsection{\texttt{shortcut() Function}}
The \texttt{shortcut()} function provides a more streamlined testing approach compared to \texttt{test()}. It is designed for quick validation and includes additional steps for error correction.

\begin{itemize}
    \item Key Reading: Unlike \texttt{test()}, the \texttt{shortcut()} function skips matrix generation and directly reads pre-existing keys from files. This makes the process faster but assumes that the keys have already been correctly generated and stored.
    \item Encryption and Decryption: The function encrypts and decrypts a random message using the pre-existing keys, similar to the \texttt{test()} function, but with less emphasis on the full lifecycle of key management.
    \item Error Correction: A significant difference is that \texttt{shortcut()} includes an error correction phase. It encrypts the message three times, reads the resulting packets, and attempts to correct any errors that may have occurred during transmission.
    \item Efficiency: The \texttt{shortcut()} function is optimized for speed and efficiency, making it ideal for situations where a quick test is needed, rather than a full validation.
\end{itemize}

The \texttt{shortcut()} function is useful for quickly verifying that the core cryptographic operations are functioning as intended, particularly in scenarios where error correction is critical. However, it does not perform the full range of checks that the \texttt{test()} function does.

\section{\texttt{api.c}}

The \texttt{api.c} file provides the core API functions for the cryptographic system, including key generation, encryption, decryption, and error correction. These functions are responsible for the main cryptographic operations and serve as the interface between the higher-level application logic and the lower-level data handling provided by the backend.

\subsection{Key Functions}

\paragraph{\texttt{generate\_key()}}
generates the cryptographic keys necessary for the encryption and decryption processes. It creates three matrices: \texttt{a} (public key matrix), \texttt{s} (private key matrix), and \texttt{e} (error matrix). The function then computes a fourth matrix, \texttt{y}, which is derived from \texttt{a}, \texttt{s}, and \texttt{e}. These keys are then written to files using the backend's \texttt{write\_key()} function.

\paragraph{\texttt{encrypt(const char *mex, const char *a\_path, const char *y\_path)}}
handles the encryption of a message. It reads the public keys \texttt{a} and \texttt{y} from the specified file paths and then encrypts the provided message \texttt{mex}. The encryption process involves generating a random error array \texttt{e}, calculating the \texttt{nnc} and \texttt{word} arrays through matrix multiplication and XOR operations, and then writing these arrays to files using the \texttt{write\_packet()} function.

\paragraph{\texttt{decrypt(const char *fnnc, const char *fword, const char *key\_path)}}
decrypts a message by reversing the encryption process. It reads the private key \texttt{s}, as well as the \texttt{nnc} and \texttt{word} arrays from the specified file paths. The decryption involves matrix multiplication and XOR operations to retrieve the original message. The decrypted message is then written to a file.

\paragraph{\texttt{correct(const char *path1, const char *path2, const char *path3)}}
performs error correction on three encrypted data packets. It reads the three packets from the specified paths and applies an error correction algorithm using the \texttt{correct\_errors()} function from the backend. The corrected data is then written to a file.

\section{\texttt{backend.c}}

The \texttt{backend.c} file provides essential input/output operations for handling cryptographic data structures, such as matrices and arrays. It includes functions for reading and writing keys, converting data formats, and correcting errors in transmitted packets. This file acts as the backbone of data handling within the cryptographic system, ensuring that data is correctly stored, retrieved, and processed.

\subsection{Key Functions}

\paragraph{\texttt{write\_key(const char *path, struct mat *m)}}
writes a matrix (used as a cryptographic key) to a binary file. It stores the matrix's dimensions (rows and columns) followed by its data, row by row.

\paragraph{\texttt{read\_key(const char *path)}}
reads a matrix from a binary file and reconstructs it into memory. It is the counterpart to \texttt{write\_key()}, enabling the cryptographic system to retrieve previously stored keys and use them for operations such as encryption, decryption, and validation.

\paragraph{\texttt{convert\_to\_array(const char *filepath)}}
converts a text file of binary data (represented as '1's and '0's) into an array of 64-bit unsigned integers. This conversion is essential for preparing data in the correct format for encryption or other cryptographic processes. The resulting array can be used directly in cryptographic operations, ensuring efficient processing of binary data.

\paragraph{\texttt{write\_packet(const char *output\_path, struct arr *message)}}
writes a data packet (an array of data) to a binary file. It first writes the length of the array and then the data itself. This function is used to store encrypted messages or any other arrays that need to be transmitted or stored for later use.

\paragraph{\texttt{read\_packet(const char *path)}}
reads a data packet from a binary file, reconstructing the array in memory. It is typically used to retrieve encrypted messages or other data structures that were previously stored using \texttt{write\_packet()}.

\paragraph{\texttt{correct\_errors(struct arr *a, struct arr *b, struct arr *c)}}
performs error correction on three arrays by applying a bitwise operation to correct errors that may have occurred during transmission. It combines the data from three sources and resolves differences to produce a corrected array.

\section{Launching the Program}

To launch the program, follow these steps. The process involves setting up the required dependencies, building the project using CMake, and running the compiled executable.

\subsection{Install \texttt{librandombytes}}
The program depends on the \texttt{librandombytes} library, which needs to be downloaded and installed before building the project from the source alreay present in the paper.

\subsection{Build the Project using CMake}
Once \texttt{librandombytes} is installed, you can build the project using CMake. The \texttt{CMakeLists.txt} file provided in the project directory contains all the necessary instructions.

\begin{itemize}
    \item Navigate to the root directory of the project.
    \item Create a build directory and navigate into it:
\begin{verbatim}
$ mkdir build
$ cd build
\end{verbatim}
    \item Run CMake to generate the makefiles:
\begin{verbatim}
$ cmake ..
\end{verbatim}
    \item Build the project using make:
\begin{verbatim}
$ make
\end{verbatim}
\end{itemize}

\subsection{Run the Program}
After successfully building the project, you can run the compiled executable.

\begin{itemize}
    \item The executable will be located in the \texttt{build/bin} directory. To run the program, use the following command:
\begin{verbatim}
$ ./your_program_name <command> [<args>]
\end{verbatim}
    \item Replace \texttt{your\_program\_name} with the actual name of the executable generated by the build process.
    \item The available commands are those described in the \texttt{main.c} section.
\end{itemize}

\subsection{Verifying the Setup}
To ensure everything is set up correctly, you can run a test command such as:
\begin{verbatim}
$ ./your_program_name test
\end{verbatim}
This command will initiate the testing process, verifying that the encryption, decryption, and key generation functionalities are working as expected.
